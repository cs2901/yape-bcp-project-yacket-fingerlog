\documentclass{article}
\usepackage[utf8]{inputenc}

\title{Document requirements of the product}
\author{jose.chavez@utec.edu.pe \\ sergio.carbone@utec.edu.pe\\antonio.toche@utec.edu.pe\\jeffrey.orihuela@utec.edu.pe\\fernando.socualaya@utec.edu.pe}
\date{April 2019}

\begin{document}

\maketitle

\section{Target user profile and value proposition}
   In our project, we have two main objective users. The first is a user of YAPE that usually goes to MYPES and PYMES(micro and small business). The other is a YAPE user owner of a MYPES or PYMES(micro and small business).\\ Both have a big liking of technology and constantly look for new ways of digitizing all the process of selling taking advantage of these tools that streamline, secure and facilitate the shopping and sales.\\
   The main value proposal from our team is to add the feature of make possible to emit a ticket or invoice when you pay through YAPE instantly.
    \subsection{User Stories} 
    \begin{itemize}
    \item As a business owner, I can organize all the tickets so that makes easier the procedures with SUNAT
    \item As a client, I can pay without cash so I won't receive fake money.
    \item As a cashier, I can generate electronic tickets/invoices, so I don't have to fill them manually.
    \item As a cashier, I can recieve payments from more than one business, so I don't have to use different smartphones. 
    \end{itemize}
    
\section{Use cases}

\begin{enumerate}

    \item As a cashier, I can recieve payments from more than one business, so I don't have to use different smartphones. 
    \begin{itemize}
        \item \textbf{Software System :} Yape 
        \item \textbf{Actor: } Cashier
        \item \textbf{Use Case : } UC01 Add business info
        \begin{enumerate}
            \item Cashier selects Menu.
            \item Cashier selects invoicing.
            \item Cashier selects the '+' symbol.
            \item Cashier fills the required info about the business.
            \item Cashier selects save.\\
            Use Case ends.
        \end{enumerate}
    \end{itemize}

    \item As a cashier, I can generate electronic tickets/invoices, so I don't have to fill them manually. 
    \begin{itemize}
        \item \textbf{Software System :} Yape
        \item \textbf{Actor: } Cashier
        \item \textbf{Use Case : } UC02 Generate QR for ticket
        \item \textbf{Preconditions: } UC01
	\begin{enumerate}
            \item Cashier selects to receive payment.
            \item Cashier selects Business.
            \item Cashier selects the ticket option.
            \item Cashier fills the items and amounts correspondent.
            \item Cashier select save and receive payment.
            \item Yape displays the generated QR\\
            Use Case ends.
        \end{enumerate}
    \end{itemize}
    
    
   \item  As a client, I can pay without cash so I won't receive fake money. 
   \begin{itemize}
        \item \textbf{Software System : } Yape 
        \item \textbf{Actor: } Consumer 
        \item \textbf{Use Case : } UC03 Pay scanning QR
        \item \textbf{Preconditions: } UC02
	\begin{enumerate}
            \item Consumer logs into Yape.
	    \item Consumer selects pay with QR code.
	    \item Consumer scan the QR generated by the cashier in UC02.
	    \item Consumer confirms the payment.\\
            Use Case ends.
        \end{enumerate}
    \end{itemize}

   \item As a Consumer/Business owner I can organize my tickets/invoices.
   \begin{itemize}
        \item \textbf{Software System :} Yape 
        \item \textbf{Actor: } Consumer, Business owner
        \item \textbf{Use Case : } Receive ticket/invoice via email
        \item \textbf{Preconditions : } UC03
	\begin{enumerate}
            \item Yape sends the ticket/invoice to the registered mails of both, consumer and business owner.
            Use Case ends.
       \end{enumerate}
    \end{itemize}
    
%    \item .
%   \begin{itemize}
%        \item \textbf{Software System :} 
%        \item \textbf{Actor: } 
%       \item \textbf{Use Case : } 
%        \begin{enumerate}
%            \item .\\
%            Use Case ends.
%        \end{enumerate}
%    \end{itemize}
    
\end{enumerate}

\section{Non-functional requirements}
\begin{enumerate}
     \item Persist the procedure with SUNAT.
     \item Good internet connection.
    
\end{enumerate}

\section{Glossary}

\begin{itemize}
    \item MYPE/PYME: Micro and Small sized enterprises.
    \item Micro sized entreprise: Conformed by less than 10 employees.
    \item Small sized entreprise: Conformed by less than 20 employess.
    \item Consumer: The person who purchase a product.
    \item MYPE/PYME owner: The person who owns/manages a MYPE/PYME.
    \item Ticket/invoice: Payment vouchers.
    \item SUNAT: Entity that collects taxes.
\end{itemize}


\section{Product survey}
A product very similar to YAPE exists on Interbank. This product is called TUNKI and so as YAPE also allows you to make payments using only a phone number.
Both have the same functions of making payments and collections from the
same application to your contacts or scanning a QR code. In the case of
companies, they can also make payments to these through the scanner a
QR code. Although both applications have the same functionalities,
YAPE for its benefit is more recognized in the market and has had a better
diffusion. However, none of the applications has anything similar to
YACKET and include as part of YAPE could give a clear added value
for this innovative tool.

\end{document}


