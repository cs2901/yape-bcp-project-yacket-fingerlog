\documentclass{article}
\usepackage[utf8]{inputenc}

\title{Document requirements of the product}
\author{jose.chavez@utec.edu.pe \\ sergio.carbone@utec.edu.pe\\antonio.toche@utec.edu.pe\\jeffrey.orihuela@utec.edu.pe\\fernando.socualaya@utec.edu.pe}
\date{April 2019}

\begin{document}

\maketitle

\section{Target user profile and Value proposition}
    EN nuestro proyecto tenemos dos usuarios objetivos . El primero es el \textbf{ usuario de Yape que frecuenta MYPES y PYMES} (micro y pequeñas empresas ). El otro es el \textbf{usuario de Yape dueno de MYPES o PYMES}. Ambos con una gran aficion a la tecnologia y en busca de nuevas formas de digitalizar los procesos de venta aprovechando estas herramientas que agilizan , aseguran y facilitan sus ventas o compras.\\ La principal propuesta de valor de Yacket es añadir la una opcion que te permita al momento de recibir un pago, hacerle llegar al remitente una boleta o factura correspondiente.     
    \begin{itemize}
    \item Como dueño de una mype, quiero que me organicen las boletas para que se me faciliten los trámites con la SUNAT.
    \item Como dueño de una mype, me gustaria recibir informacion sobre las ventas a lo largo de los dias de la semana para poder distribuir mejor los recursos.
    \item Como cliente, quiero poder ver organizados mis gastos para así poder invertir mejor mi dinero y no hacer gastos innecesarios.
    \item Como cliente, quiero poder pagar sin efectivo para no recibir dinero falso.
    \item Como cliente me gustaria que se pueda dividir la cuenta entre varias personas para poder facilitar el pago



    \end{itemize}
    
\section{Use cases}
\begin{enumerate}
    \item Como dueño de una mype, me gustaría guardar un registro de mis clientes para poder reconocerlos y fidelizarlos.
    \begin{itemize}
        \item \textbf{Software System :} Yape
        \item \textbf{Actor: } Dueño MYPE , Cliente
        \item \textbf{Use Case : } Guardar clientes frecuentes 
        \begin{enumerate}
            \item Cliente realiza pago
            \item Dueño MYPE recibe pago de Cliente
            \item Yape guarda contador de transacciones de Cliente
            \item Sumarle uno a contador de Cliente
            \item Actualizar contadores
            \item Listar Clientes con mas transacciones\\
            Use Case ends.
        \end{enumerate}
    \end{itemize}
    
    \item Como cliente quiero contar con boletas o facturas electronicas para asi tener la seguridad de que nunca se me perdera alguna.
   \begin{itemize}
        \item \textbf{Software System :} Yape 
        \item \textbf{Actor: } Cliente
        \item \textbf{Use Case : } Generar Boleta/Factura electronica 
        \begin{enumerate}
            \item Cliente escanea codigo QR de transaccion
            \item Yape recopila los datos de transaccion
            \item Yape hace un request a los servidores de la SUNAT
            \item Yape genera y envia el pdf al Cliente
            \item Yape espera la respuesta de la BDD de SUNAT\\
            Use Case ends.
        \end{enumerate}
    \end{itemize}
    
    \item .
   \begin{itemize}
        \item \textbf{Software System :} 
        \item \textbf{Actor: } 
        \item \textbf{Use Case : } 
        \begin{enumerate}
            \item .\\
            Use Case ends.
        \end{enumerate}
    \end{itemize}


 \item .
   \begin{itemize}
        \item \textbf{Software System :} 
        \item \textbf{Actor: } 
        \item \textbf{Use Case : } 
        \begin{enumerate}
            \item .\\
            Use Case ends.
        \end{enumerate}
    \end{itemize}
    
    \item .
   \begin{itemize}
        \item \textbf{Software System :} 
        \item \textbf{Actor: } 
        \item \textbf{Use Case : } 
        \begin{enumerate}
            \item .\\
            Use Case ends.
        \end{enumerate}
    \end{itemize}
    
\end{enumerate}

\section{Non-functional requirements}
\begin{enumerate}
    \item Poder organizar segun sector los pagos que realiza cada usuario.
    \item LLevar un registro de los consumidores y se analize los dias mas concurrentes y de mayor demanda.
    \item Aceptar dentro de el pago propina o estar en la capacidad de dividar la cuenta en un numero determinado de personas.
    \item Poder soportar pagos de pedidos a domicilio.
\end{enumerate}

\section{Glossary}



\section{Product survey}
Un producto muy similar a YAPE existe por parte del banco interbank. Este producto se llama TUNKI y asi como YAPE tambien permite hacer pagos de dinero usando solamente un numero telefonico.\\
Ambos tienen las mismas fucionalidades de realizar pagos y cobros desde la misma applicacion a sus contactos o escaneando un codigo QR. En el caso de las empresas tambien se le puede realizar pagos a estas escaneando un codigo QR. Si bien ambas aplicaciones presentan las mismas funcionalidades YAPE en su beneficio es mas reconocida en el mercado y ah tenido una mejor difucion.  Sin embargo ninfuna de las aplicaciones cuenta con algo similar a YACKET e incluir como parte de YAPE podria darle un claro valor agregado por esta innovadora herramieta.



\end{document}

